%        File: siip.tex
%     Created: Mon Feb 27 08:00 AM 2017 C
% Last Change: Mon Feb 27 08:00 AM 2017 C
%
\documentclass[11pt]{article}

%\usepackage[acronym,toc]{glossaries}
%%\newacronym{<++>}{<++>}{<++>}
\newacronym[longplural={metric tons of heavy metal}]{MTHM}{MTHM}{metric ton of heavy metal}
\newacronym{ABM}{ABM}{agent-based modeling}
\newacronym{ACDIS}{ACDIS}{Program in Arms Control \& Domestic and International Security}
\newacronym{ADS}{ADS}{Accelerator-Driven System}
\newacronym{AHTR}{AHTR}{Advanced High Temperature Reactor}
\newacronym{ANDRA}{ANDRA}{Agence Nationale pour la gestion des D\'echets RAdioactifs, the French National Agency for Radioactive Waste Management}
\newacronym{ANL}{ANL}{Argonne National Laboratory}
\newacronym{ANS}{ANS}{American Nuclear Society}
\newacronym{API}{API}{application programming interface}
\newacronym{ARE}{ARE}{Aircraft Reactor Experiment}
\newacronym{ARFC}{ARFC}{Advanced Reactors and Fuel Cycles}
\newacronym{ASME}{ASME}{American Society of Mechanical Engineers}
\newacronym{ASTRID}{ASTRID}{Advanced Sodium Technological Reactor for Industrial Demonstration}
\newacronym{ATWS}{ATWS}{Anticipated Transient Without Scram}
\newacronym{BDBE}{BDBE}{Beyond Design Basis Event}
\newacronym{BIDS}{BIDS}{Berkeley Institute for Data Science}
\newacronym{BWR}{BWR}{Boiling Water Reactor}
\newacronym{CAFCA}{CAFCA}{ Code for Advanced Fuel Cycles Assessment }
\newacronym{CDTN}{CDTN}{Centro de Desenvolvimento da Tecnologia Nuclear}
\newacronym{CEA}{CEA}{Commissariat \`a l'\'Energie Atomique et aux \'Energies Alternatives}
\newacronym{CI}{CI}{continuous integration}
\newacronym{CNEN}{CNEN}{Comiss\~{a}o Nacional de Energia Nuclear}
\newacronym{CNERG}{CNERG}{Computational Nuclear Engineering Research Group}
\newacronym{COSI}{COSI}{Commelini-Sicard}
\newacronym{COTS}{COTS}{commercial, off-the-shelf}
\newacronym{CSNF}{CSNF}{commercial spent nuclear fuel}
\newacronym{CTAH}{CTAHs}{Coiled Tube Air Heaters}
\newacronym{CUBIT}{CUBIT}{CUBIT Geometry and Mesh Generation Toolkit}
\newacronym{CURIE}{CURIE}{Centralized Used Fuel Resource for Information Exchange}
\newacronym{DAG}{DAG}{directed acyclic graph}
\newacronym{DANESS}{DANESS}{Dynamic Analysis of Nuclear Energy System Strategies}
\newacronym{DBE}{DBE}{Design Basis Event}
\newacronym{DESAE}{DESAE}{Dynamic Analysis of Nuclear Energy Systems Strategies}
\newacronym{DHS}{DHS}{Department of Homeland Security}
\newacronym{DOE}{DOE}{Department of Energy}
\newacronym{DRACS}{DRACS}{Direct Reactor Auxiliary Cooling System}
\newacronym{DRE}{DRE}{dynamic resource exchange}
\newacronym{DSNF}{DSNF}{DOE spent nuclear fuel}
\newacronym{DYMOND}{DYMOND}{Dynamic Model of Nuclear Development }
\newacronym{EBS}{EBS}{Engineered Barrier System}
\newacronym{EDF}{EDF}{Électricité de France}
\newacronym{EDZ}{EDZ}{Excavation Disturbed Zone}
\newacronym{EIA}{EIA}{U.S. Energy Information Administration}
\newacronym{EPA}{EPA}{Environmental Protection Agency}
\newacronym{EPR}{EPR}{European Pressurized Reactor}
\newacronym{EP}{EP}{Engineering Physics}
\newacronym{EU}{EU}{European Union}
\newacronym{FCO}{FCO}{Fuel Cycle Options}
\newacronym{FCT}{FCT}{Fuel Cycle Technology}
\newacronym{FEHM}{FEHM}{Finite Element Heat and Mass Transfer}
\newacronym{FEPs}{FEPs}{Features, Events, and Processes}
\newacronym{FHR}{FHR}{Fluoride-Salt-Cooled High-Temperature Reactor}
\newacronym{FLiBe}{FLiBe}{Fluoride-Lithium-Beryllium}
\newacronym{FP}{FP}{Fission Products}
\newacronym{GDSE}{GDSE}{Generic Disposal System Environment}
\newacronym{GDSM}{GDSM}{Generic Disposal System Model}
\newacronym{GENIUSv1}{GENIUSv1}{Global Evaluation of Nuclear Infrastructure Utilization Scenarios, Version 1}
\newacronym{GENIUSv2}{GENIUSv2}{Global Evaluation of Nuclear Infrastructure Utilization Scenarios, Version 2}
\newacronym{GENIUS}{GENIUS}{Global Evaluation of Nuclear Infrastructure Utilization Scenarios}
\newacronym{GPAM}{GPAM}{Generic Performance Assessment Model}
\newacronym{GRSAC}{GRSAC}{Graphite Reactor Severe Accident Code}
\newacronym{GUI}{GUI}{graphical user interface}
\newacronym{GWe}{GWe}{gigawatts electric}
\newacronym{HLW}{HLW}{high level waste}
\newacronym{HPC}{HPC}{high-performance computing}
\newacronym{HTC}{HTC}{high-throughput computing}
\newacronym{HTGR}{HTGR}{High Temperature Gas-Cooled Reactor}
\newacronym{IAEA}{IAEA}{International Atomic Energy Agency}
\newacronym{IEMA}{IEMA}{Illinois Emergency Mangament Agency}
\newacronym{IHLRWM}{IHLRWM}{International High Level Radioactive Waste Management}
\newacronym{INL}{INL}{Idaho National Laboratory}
\newacronym{IPRR1}{IRP-R1}{Instituto de Pesquisas Radioativas Reator 1}
\newacronym{IRP}{IRP}{Integrated Research Project}
\newacronym{ISFSI}{ISFSI}{Independent Spent Fuel Storage Installation}
\newacronym{ISRG}{ISRG}{Independent Student Research Group}
\newacronym{JFNK}{JFNK}{Jacobian-Free Newton Krylov}
\newacronym{LANL}{LANL}{Los Alamos National Laboratory}
\newacronym{LBNL}{LBNL}{Lawrence Berkeley National Laboratory}
\newacronym{LCOE}{LCOE}{levelized cost of electricity}
\newacronym{LDRD}{LDRD}{laboratory directed research and development}
\newacronym{LFR}{LFR}{Lead-Cooled Fast Reactor}
\newacronym{LLNL}{LLNL}{Lawrence Livermore National Laboratory}
\newacronym{LMFBR}{LMFBR}{Liquid Metal Fast Breeder Reactor}
\newacronym{LOFC}{LOFC}{Loss of Forced Cooling}
\newacronym{LOHS}{LOHS}{Loss of Heat Sink}
\newacronym{LOLA}{LOLA}{Loss of Large Area}
\newacronym{LP}{LP}{linear program}
\newacronym{LWR}{LWR}{Light Water Reactor}
\newacronym{MAGNOX}{MAGNOX}{Magnesium Alloy Graphie Moderated Gas Cooled Uranium Oxide Reactor}
\newacronym{MA}{MA}{minor actinide}
\newacronym{MCNP}{MCNP}{Monte Carlo N-Particle code}
\newacronym{MILP}{MILP}{mixed-integer linear program}
\newacronym{MIT}{MIT}{the Massachusetts Institute of Technology}
\newacronym{MOAB}{MOAB}{Mesh-Oriented datABase}
\newacronym{MOOSE}{MOOSE}{Multiphysics Object-Oriented Simulation Environment}
\newacronym{MOX}{MOX}{Mixed Oxide Fuel}
\newacronym{MSBR}{MSBR}{Molten Salt Breeder Reactor}
\newacronym{MSRE}{MSRE}{Molten Salt Reactor Experiment}
\newacronym{MSR}{MSR}{Molten Salt Reactor}
\newacronym{MWe}{MWe}{megawatts electric}
\newacronym{NAGRA}{NAGRA}{National Cooperative for the Disposal of Radioactive Waste}
\newacronym{NEAMS}{NEAMS}{Nuclear Engineering Advanced Modeling and Simulation}
\newacronym{NEUP}{NEUP}{Nuclear Energy University Programs}
\newacronym{NFCSim}{NFCSim}{Nuclear Fuel Cycle Simulator}
\newacronym{NGNP}{NGNP}{Next Generation Nuclear Plant}
\newacronym{NMWPC}{NMWPC}{Nuclear MW Per Capita}
\newacronym{NNSA}{NNSA}{National Nuclear Security Administration}
\newacronym{NPRE}{NPRE}{Department of Nuclear, Plasma, and Radiological Engineering}
\newacronym{NQA1}{NQA-1}{Nuclear Quality Assurance - 1}
\newacronym{NRC}{NRC}{Nuclear Regulatory Commission}
\newacronym{NSF}{NSF}{National Science Foundation}
\newacronym{NSSC}{NSSC}{Nuclear Science and Security Consortium}
\newacronym{NUWASTE}{NUWASTE}{Nuclear Waste Assessment System for Technical Evaluation}
\newacronym{NWF}{NWF}{Nuclear Waste Fund}
\newacronym{NWTRB}{NWTRB}{Nuclear Waste Technical Review Board}
\newacronym{OCRWM}{OCRWM}{Office of Civilian Radioactive Waste Management}
\newacronym{ORION}{ORION}{ORION}
\newacronym{ORNL}{ORNL}{Oak Ridge National Laboratory}
\newacronym{PARCS}{PARCS}{Purdue Advanced Reactor Core Simulator}
\newacronym{PBAHTR}{PB-AHTR}{Pebble Bed Advanced High Temperature Reactor}
\newacronym{PBFHR}{PB-FHR}{Pebble-Bed Fluoride-Salt-Cooled High-Temperature Reactor}
\newacronym{PEI}{PEI}{Peak Environmental Impact}
\newacronym{PHWR}{Pressurized Heavy Water Reactor}{Pressurized Heavy Water Reactor}
\newacronym{PH}{PRONGHORN}{PRONGHORN}
\newacronym{PRIS}{PRIS}{Power Reactor Information System}
\newacronym{PRKE}{PRKE}{Point Reactor Kinetics Equations}
\newacronym{PSPG}{PSPG}{Pressure-Stabilizing/Petrov-Galerkin}
\newacronym{PWAR}{PWAR}{Pratt and Whitney Aircraft Reactor}
\newacronym{PWR}{PWR}{Pressurized Water Reactor}
\newacronym{PyNE}{PyNE}{Python toolkit for Nuclear Engineering}
\newacronym{PyRK}{PyRK}{Python for Reactor Kinetics}
\newacronym{QA}{QA}{quality assurance}
\newacronym{RDD}{RD\&D}{Research Development and Demonstration}
\newacronym{RD}{R\&D}{Research and Development}
\newacronym{RELAP}{RELAP}{Reactor Excursion and Leak Analysis Program}
\newacronym{RIA}{RIA}{Reactivity Insertion Accident}
\newacronym{RIF}{RIF}{Region-Institution-Facility}
\newacronym{SFR}{SFR}{Sodium-Cooled Fast Reactor}
\newacronym{SINDAG}{SINDA{\textbackslash}G}{Systems Improved Numerical Differencing Analyzer $\backslash$ Gaski}
\newacronym{SKB}{SKB}{Svensk K\"{a}rnbr\"{a}nslehantering AB}
\newacronym{SNF}{SNF}{spent nuclear fuel}
\newacronym{SNL}{SNL}{Sandia National Laboratory}
\newacronym{STC}{STC}{specific temperature change}
\newacronym{SUPG}{SUPG}{Streamline-Upwind/Petrov-Galerkin}
\newacronym{SWF}{SWF}{Separations and Waste Forms}
\newacronym{SWU}{SWU}{Separative Work Unit}
\newacronym{TRIGA}{TRIGA}{Training Research Isotope General Atomic}
\newacronym{TRISO}{TRISO}{Tristructural Isotropic}
\newacronym{TSM}{TSM}{Total System Model}
\newacronym{TSPA}{TSPA}{Total System Performance Assessment for the Yucca Mountain License Application}
\newacronym{ThOX}{ThOX}{thorium oxide}
\newacronym{UFD}{UFD}{Used Fuel Disposition}
\newacronym{UML}{UML}{Unified Modeling Language}
\newacronym{UNF}{UNF}{Used Nuclear Fuel}
\newacronym{UOX}{UOX}{Uranium Oxide Fuel}
\newacronym{UQ}{UQ}{uncertainty quantification}
\newacronym{US}{US}{United States}
\newacronym{UW}{UW}{University of Wisconsin}
\newacronym{VISION}{VISION}{the Verifiable Fuel Cycle Simulation Model}
\newacronym{VVER}{VVER}{Voda-Vodyanoi Energetichesky Reaktor (Russian Pressurized Water Reactor)}
\newacronym{VV}{V\&V}{verification and validation}
\newacronym{WIPP}{WIPP}{Waste Isolation Pilot Plant}
\newacronym{YMR}{YMR}{Yucca Mountain Repository Site}

%\makeglossaries
\usepackage{fancyhdr}
\usepackage{pagecounting}
\usepackage[dvips]{color}
\usepackage{graphicx}
\usepackage{caption}
\usepackage{subcaption}
\usepackage{placeins}
\usepackage[hidelinks]{hyperref}
\usepackage{tabularx}

% Trying to bold PI names in the bib
\usepackage{xstring}
\def\FormatName#1{%
          \IfSubStr{#1}{Huff}{\textbf{#1}}{\IfSubStr{#1}{Neal Davis}{\textbf{#1}}{#1}}%
          }

          \usepackage[left=1in, right=1in, top=1in, bottom=1in]{geometry}
          \newcommand\bb[1]{\mbox{\em #1}}
          \def\baselinestretch{1.1}
          %\pagestyle{empty}
          \newcommand{\hsp}{\hspace*{\parindent}}
          \definecolor{gray}{rgb}{0.4,0.4,0.4}

          \newcommand{\authorname}{Kathryn~D.~Huff }
          \newcommand{\authoremail}{kdhuff@illinois.edu}
          \newcommand{\authorsite}{arfc.npre.illinois.edu}

          \begin{document}
          \title{Collaborative Open Source Curriculum Development}
          \author{\textbf{PI: Kathryn Huff}\\
                  kdhuff@illinois.edu\\
                  University of Illinois Urbana-Champaign
                  \and
\textbf{Co-PI: Neal Davis}\\davis68@illinois.edu\\University of Illinois Urbana-Champaign
\and
Paul Wilson\\University of Wisconsin - Madison 
\and
          Steven Skutnik\\University of Tennessee - Knoxville
\and
          Anthony Scopatz\\University of South Carolina 
\and
          Jeremy Roberts\\Kansas State University 
\and
          Robert Borrelli\\University of Idaho at Idaho Falls
}
          \maketitle

          \pagestyle{fancy}
          %\pagenumbering{gobble}
          %\fancyhead[location]{text}
          % Leave Left and Right Header empty.
          %\lhead{}
          %\rhead{}
          \lhead{\textcolor{gray}{SIIP Close-Out Report}}
          \rhead{\textcolor{gray}{Collaborative Open Source Curriculum Development}}
          %\rhead{\textcolor{gray}{\thepage/\totalpages{}}}
          \renewcommand{\headrulewidth}{0pt}
          \renewcommand{\footrulewidth}{0pt}
          %\fancyfoot[C]{\footnotesize \textcolor{gray}{\authorsite}}

          \section{Goals Met and Outcomes Acheived}
          Our original goals and deliverables were to establish a small-scale 
          proof-of-concept for collaborative, open source, curriculum 
          development to improve the transfer of lessons learned between 
          instructors of the same course (either at a single university or 
          among different campuses). This prototype collaboration will provide 
          a template which could be adopted for collaboration among faculty 
          teaching courses with an inherently larger scale (e.g.  CS101).

          This group successfully:
          \begin{itemize}
                  \item Held a kick-off workshop 
                  \item Held weekly Google Hangouts
                  \item Established communication channels through GitHub's 
                          collaboration framework, a Slack channel, and a 
                          shared Google Drive folder.
                  \item Created a template 
                          collaborative, co-owned model of shared work on 
                          shared curriculum. We call it ``The Nuclear 
                          Engineering Curriculum eXchange (NECX)''.
                  \item Created a GitHub organization which can be found
                          at https://github.com/necx-org. 
                  \item Conceived and documented processes for the collaboration in that repository.
                  \item Created a website where the product is being rendered https://necx-org.github.io/.
                  \item Developed a definition of a single unit of learning, a 
                          `node'.
                  \item Established acceptable raw formats for node 
                          contributions.
                  \item Developed requirements and workflow for contributing a `node'
                  \item Developed review criteria for reviewing node contributions.
                  \item Began to add nodes https://github.com/necx-org/nodes 
                          toward a master set of learning modules for an 
                          upper-division course in nuclear engineering : The 
                          Nuclear Fuel Cycle. These were submitted by pull 
                          request to the main repository and each includes:
          \begin{itemize} 
                  \item associated learning objectives (identified previously)
                \item content (e.g. speaking notes, presentation material, 
                        derivations, worked examples, active learning  
                          exercises, external readings, videos, images)
                  \item learning assessments (e.g. project descriptions, exam questions) 
          \end{itemize}
                  \item Shared insight with one another that couldn’t have been found at our home institutions.
                  \item Brought these lessons into our classrooms, particularly 
                          a modular perspective on learning outcomes and assessments.
          \end{itemize}

          \section{Expenses}
          The total budget for the work was \$19,347. \textbf{Please note:} The 
          project year is bookended by two workshops, a kickoff workshop 
          (already occured in July 2017) and a retrospective workshop 
          (scheduled for July 2018). 

          \begin{itemize}
                  \item \textbf{Jul 2017} Kick-Off Workshop
                  \item \textbf{Aug 2017 - Dec 2017} NPRE 412 beta testing
                  \item \textbf{Jul 2018} Retrospective Workshop
          \end{itemize}

          
          Accordingly, the expenses for the Kickoff Workshop and the Summer Salary for Co-PI 
          Davis have already been spent. \textbf{The expenses for the Retrospective 
  workshop will be spent in July 2018.} Table \ref{tab:budget} provides details related to the expenses for 
          this effort.

\begin{table}[h!]
        \begin{tabularx}{\textwidth}{|X|c|c|c|}
        \hline
        \textbf{Event} & \textbf{Item} & \textbf{Cost}  & \textbf{Notes}\\
        \hline
Kick-off Workshop &Meeting Room, Lodging, \& Food &\$2,778.74&Allerton Meeting Fee \\ 
&Dinner 1 &\$244.50&Radio Maria\\
&Dinner 2 &\$131.63&Destihl\\
\hline
Lessons Learned Report & Action Research & \$7225.00 & Neal Davis Summer Salary\\
Celebration of Teaching & Poster Printing & \$100.00 & printing.illinois.edu\\
\hline
Retrospective Workshop&Meeting Room&100&Illini Union\\
&Projector&1200&Illini Union\\
&Lodging&3600&Illini Union Hotel\\
&Continental breakfast&204&University Catering\\
&Coffee Service&130&University Catering\\
&Lunches &360&University Catering\\
&Dinner 1&300&In Champaign-Urbana\\
&Dinner 2&600&In Champaign-Urbana\\
        \hline
        &\textbf{Total}&\textbf{19073.87}&\\
        \hline
\end{tabularx}
\caption{Expenses supporting workshop necessities and reporting activities.}
\label{tab:budget}
\end{table}


          \section{Best Of}
          Much of the work focused on development of processes that could 
          sustainably support a community of dispersed faculty collaborating on 
          shared, modular curriculum. A key example of this is the 
          Contributions document we collectively created which acts as a 
          linchpin process document. This document, Making Contributions to the Nuclear 
          Engineering Curriculum eXchange, can be found at 
          \url{https://github.com/necx-org/nodes/blob/master/CONTRIBUTING.md} . 
          This document is an excellent template for future groups thiking of 
          building a similar community and captures the most essential of the 
          processes we established for this group.

          \section{Publicizing}
          We publicized this project through:

          \begin{itemize}
                  \item Inviting NPRE department colleagues to the kickoff 
                          workshop dinner.
                  \item The github organization
                  \item The website
                  \item Poster for the Celebration of Teaching
          \end{itemize}
          Additionally, the retrospective workshop will present an opportunity 
          to invite NPRE colleagues and AE3 colleagues to a lessons learned 
          presentation.
          
          \section{Departmental Support}
          The Department of Nuclear, Plasma, and Radiological Engineering
          support this work in kind with release time for Kathryn Huff through 
          her Start-Up funds.  Additionally, NPRE supported workshop activities 
          with administrative effort and by providing space. 


          \bibliographystyle{katyunsrt}
          \bibliography{2017-siip}


          \end{document}


