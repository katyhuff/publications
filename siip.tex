%        File: siip.tex
%     Created: Mon Feb 27 08:00 AM 2017 C
% Last Change: Mon Feb 27 08:00 AM 2017 C
%
\documentclass[11pt]{article}

%\usepackage[acronym,toc]{glossaries}
%\include{acros}
%\makeglossaries
\usepackage{fancyhdr}
\usepackage{pagecounting}
\usepackage[dvips]{color}
\usepackage{graphicx}
\usepackage{caption}
\usepackage{subcaption}
\usepackage{placeins}
\usepackage[hidelinks]{hyperref}
\usepackage{tabularx}

% Trying to bold PI names in the bib
\usepackage{xstring}
\def\FormatName#1{%
          \IfSubStr{#1}{Huff}{\textbf{#1}}{\IfSubStr{#1}{Neal Davis}{\textbf{#1}}{#1}}%
          }

          \usepackage[left=1in, right=1in, top=1in, bottom=1in]{geometry}
          \newcommand\bb[1]{\mbox{\em #1}}
          \def\baselinestretch{1.1}
          %\pagestyle{empty}
          \newcommand{\hsp}{\hspace*{\parindent}}
          \definecolor{gray}{rgb}{0.4,0.4,0.4}

          \newcommand{\authorname}{Kathryn~D.~Huff }
          \newcommand{\authoremail}{kdhuff@illinois.edu}
          \newcommand{\authorsite}{arfc.npre.illinois.edu}

          \begin{document}
          \title{Collaborative Open Source Curriculum Development}
          \author{\textbf{PI: Kathryn Huff}\\
                  kdhuff@illinois.edu\\
                  University of Illinois Urbana-Champaign
                  \and
\textbf{Co-PI: Neal Davis}\\davis68@illinois.edu\\University of Illinois Urbana-Champaign
\and
Paul Wilson\\University of Wisconsin - Madison 
\and
          Steven Skutnik\\University of Tennessee - Knoxville
\and
          Anthony Scopatz\\University of South Carolina 
\and
          Jeremy Roberts\\Kansas State University 
\and
          Robert Borrelli\\University of Idaho at Idaho Falls
}
          \maketitle

          \pagestyle{fancy}
          %\pagenumbering{gobble}
          %\fancyhead[location]{text}
          % Leave Left and Right Header empty.
          %\lhead{}
          %\rhead{}
          \lhead{\textcolor{gray}{SIIP Close-Out Report}}
          \rhead{\textcolor{gray}{Collaborative Open Source Curriculum Development}}
          %\rhead{\textcolor{gray}{\thepage/\totalpages{}}}
          \renewcommand{\headrulewidth}{0pt}
          \renewcommand{\footrulewidth}{0pt}
          %\fancyfoot[C]{\footnotesize \textcolor{gray}{\authorsite}}

A final report (goals met, outcomes
achieved, total expenditures and
remaining funds)

A “Best of” short paragraph, photo, or
link that highlights a particularly
positive outcome of your project

A list of any ways you have publicized
your project (papers, faculty meetings,
websites, etc.)

A description of ways your department
has supported your project

          \section{Goals Met and Outcomes Acheived}
          Our original goals and deliverables were to establish a small-scale 
          proof-of-concept for collaborative, open source, curriculum 
          development to improve the transfer of lessons learned between 
          instructors of the same course (either at a single university or 
          among different campuses). This prototype collaboration will provide 
          a template which could be adopted for collaboration among faculty 
          teaching courses with an inherently larger scale (e.g.  CS101).

          This group successfully:
          \begin{itemize}
                  \item Held a kick-off workshop 
                  \item Held weekly Google Hangouts
                  \item Established communication channels through GitHub's 
                          collaboration framework, a Slack channel, and a 
                          shared Google Drive folder.
                  \item Created a template 
                          collaborative, co-owned model of shared work on 
                          shared curriculum. We call it ``The Nuclear 
                          Engineering Curriculum eXchange (NECX)''.
                  \item Created a GitHub organization which can be found
                          at https://github.com/necx-org. 
                  \item Conceived and documented processes for the collaboration in that repository.
                  \item Created a website where the product is being rendered https://necx-org.github.io/.
                  \item Developed a definition of a single unit of learning, a 
                          `node'.
                  \item Established acceptable raw formats for node 
                          contributions.
                  \item Developed requirements and workflow for contributing a `node'
                  \item Developed review criteria for reviewing node contributions.
                  \item Began to add nodes https://github.com/necx-org/nodes 
                          toward a master set of learning modules for an 
                          upper-division course in nuclear engineering : The 
                          Nuclear Fuel Cycle. These were submitted by pull 
                          request to the main repository and each includes:
          \begin{itemize} 
                  \item associated learning objectives (identified previously)
                \item content (e.g. speaking notes, presentation material, 
                        derivations, worked examples, active learning  
                          exercises, external readings, videos, images)
                  \item learning assessments (e.g. project descriptions, exam questions) 
          \end{itemize}
                  \item Shared insight with one another that couldn’t have been found at our home institutions.
                  \item Brought these lessons into our classrooms, particularly 
                          a modular perspective on learning outcomes and assessments.
          \end{itemize}

          \section{Expenses}
          \textbf{Note:} The project year will be bookended by two workshops, a kickoff 
          workshop (already occured in July 2017) and a retrospective workshop 
          (scheduled for July 2018). 

          \begin{itemize}
                  \item \textbf{Jul 2017} Kick-Off Workshop
                  \item \textbf{Aug 2017 - Dec 2017} NPRE 412 beta testing
                  \item \textbf{Jul 2018} Retrospective Workshop
          \end{itemize}

          
          Accordingly, the expenses for the Kickoff Workshop and the Summer Salary for Co-PI 
          Davis have already been spent. The expenses for the Retrospective 
          workshop will be spent in July 2018.

          Table \ref{tab:budget} provides details related to the expenses for 
          this effort.

\begin{table}[h!]
        \begin{tabularx}{\textwidth}{|X|c|c|c|c|c|}
        \hline
        \textbf{Event} & \textbf{Item} & \textbf{Cost} & \textbf{Units} & \textbf{\$} & \textbf{Notes}\\ 
        \hline
Kick-off Workshop&Meeting Room&744&2&1488&Allerton, all day package\\
&Lodging&150&24&3600&Allerton guest rooms\\
&Incidentals&100&6&600&\\
&Continental breakfast&8.5&24&204&Working breakfasts, Allerton\\
&Dinner&30&24&720&In Champaign-Urbana\\
\hline
Retrospective Workshop&Meeting Room&50&2&100&Illini Union\\
&Projector&300&4&1200&Illini Union\\
&Lodging&150&24&3600&Illini Union Hotel\\
&Taxi&20&12&240&\\
&Incidentals&100&6&600&\\
&Continental breakfast&8.5&24&204&University Catering\\
&Coffee Service&32.5&4&130&\\
&Lunch&15&24&360&University Catering\\
&Dinner&30&24&720&In Champaign-Urbana\\
        \hline
        &\textbf{Total}&&&\textbf{23606}&\\
        \hline
\end{tabularx}
\caption{All expenses supported workshop 
necessities.}
\label{tab:budget}
\end{table}


          \section{Best Of}
          Much of the work focused on development of processes that could 
          sustainably support a community of dispersed faculty collaborating on 
          shared, modular curriculum. A key example of this is the 
          Contributions document we collectively created which acts as a 
          linchpin process document. This document, Making Contributions to the Nuclear 
          Engineering Curriculum eXchange, can be found at 
          \url{https://github.com/necx-org/nodes/blob/master/CONTRIBUTING.md} . 
          This document is an excellent template for future groups thiking of 
          building a similar community and captures the most essential of the 
          processes we established for this group.

          \section{Publicizing}
          We publicized this project through:

          \begin{itemize}
                  \item Inviting NPRE department colleagues to the kickoff 
                          workshop dinner.
                  \item The github organization
                  \item The website
                  \item Poster for the Celebration of Teaching
          \end{itemize}
          Additionally, the retrospective workshop will present an opportunity 
          to invite NPRE colleagues and AE3 colleagues to a lessons learned 
          presentation.
          
          \section{Departmental Support}
          The Department of Nuclear, Plasma, and Radiological Engineering
          support this work in kind with release time for Kathryn Huff through 
          her Start-Up funds.  Additionally, NPRE supported workshop activities 
          with administrative effort and by providing space. 


          \bibliographystyle{katyunsrt}
          \bibliography{2017-siip}


          \end{document}


