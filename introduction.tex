
\section{Introduction}
The French 2012-2015 Commission Nationale d'Evaluation Reports
\cite{cne2_reports_2015} emphasize preparation for a transition from \glspl{LWR} to \glspl{SFR}.
However, the transition timeline is limited by the quantity and quality 
of \gls{UNF} available for reprocessing. Simultaneously, many \gls{EU} nations 
have no roadmap for disposal of \gls{UNF} accumulating within their borders.

To solve both challenges at once, this work asked whether France might fuel new 
\glspl{SFR} with \gls{MOX} generated by reprocessing the \gls{UNF} currently 
burdening its \gls{EU} partners. We concluded that the \gls{MOX} created will 
fuel French transition to a \gls{SFR} fleet and allow France to avoid building 
additional \glspl{LWR}.

We used \Cyclus \cite{huff_fundamental_2016} to answer this question. 
\Cyclus is an agent-based extensible framework for modeling the flow of 
material through future nuclear cycles.  \Cycamore is the library of agent 
models used to represent regions, institutions, and facilities. The simulations 
inthis work calculated used fuel inventory in \gls{EU} member states, modeled French 
transition to a full \gls{SFR} fleet, and evaluated this potential collaborative
strategy of used fuel management between France and its \gls{EU} partners.

French transition analyses in the current literature only consider \gls{UNF}
produced by France and require that additional \glspl{LWR}, namely \glspl{EPR},
supply the \gls{UNF} needed to meet \gls{MOX} demand \cite{carre_overview_2009,
martin_symbiotic_2017, freynet_multiobjective_2016}.  While the PArtitioning
and  Transmutation  European ROadmap  for  Sustainable nuclear  energy
(PATEROS) project \cite{fazio_study_2013} did consider fuel cycle collaboration
among \gls{EU} nations, that strategy relied on partitioning and transmutation
involving \glspl{ADS} and did not pursue a transition to \glspl{SFR} in France
\cite{fazio_study_2013}.  However, no recent work considers synergistic
international spent fuel arrangements in the \gls{EU}.  By considering France
in the context of its \gls{EU} partner nations, this work finds that a
collaborative strategy can acheive technology transition and reduce the need to
construct additional \glspl{LWR} in France.
